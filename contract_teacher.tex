\documentclass[a4paper,12pt]{article}
\usepackage[utf8]{vietnam}
\usepackage[T5]{fontenc}
\usepackage[margin=2.5cm]{geometry}
\usepackage{graphicx}
\usepackage{array}
\usepackage{booktabs}
\usepackage{colortbl}
\usepackage{xcolor}
\usepackage{fancyhdr}
\usepackage{lastpage}
\usepackage{hyperref}

% Định dạng header và footer
\pagestyle{fancy}
\fancyhf{}
\fancyhead[L]{\small\textit{Hợp đồng lao động giáo viên}}
\fancyhead[R]{\small\textit{Trường THPT Phước Long}}
\fancyfoot[C]{\thepage\ / \pageref{LastPage}}
\renewcommand{\headrulewidth}{0.4pt}
\renewcommand{\footrulewidth}{0.4pt}

% Định dạng màu sắc
\definecolor{schoolblue}{RGB}{0,84,166}
\definecolor{lightgray}{RGB}{240,240,240}

\begin{document}

% ============ HEADER ============
\thispagestyle{empty}
\begin{center}
    \textcolor{schoolblue}{\rule{\textwidth}{2pt}}
    
    \vspace{0.3cm}
    
    {\Large\textbf{TRƯỜNG TRUNG HỌC PHỔ THÔNG PHƯỚC LONG}}\\[0.2cm]
    {\large NEW YORK INTERNATIONAL SCHOOL}\\[0.2cm]
    {\normalsize Website: \href{https://newyork.io.vn}{newyork.io.vn} | Email: info@newyork.io.vn}\\
    {\normalsize Địa chỉ: Xã Phước Long, Huyện Bác Ái, Tỉnh Ninh Thuận}\\
    {\normalsize Điện thoại: 028 3731 1996}\\[0.3cm]
    
    \textcolor{schoolblue}{\rule{\textwidth}{2pt}}
\end{center}

\vspace{0.5cm}

\begin{center}
    {\LARGE\textbf{CỘNG HÒA XÃ HỘI CHỦ NGHĨA VIỆT NAM}}\\[0.2cm]
    {\large\textit{Độc lập -- Tự do -- Hạnh phúc}}\\[0.5cm]
    \rule{8cm}{0.5pt}
\end{center}

\vspace{1cm}

\begin{center}
    {\Huge\textbf{\textcolor{schoolblue}{HỢP ĐỒNG LAO ĐỘNG}}}\\[0.3cm]
    {\Large\textbf{GIÁO VIÊN}}\\[0.5cm]
    {\large Số: \textbf{156/2026/HĐGV-THPTPL}}\\[0.3cm]
    {\normalsize\textit{(Hợp đồng xác định thời hạn 12 tháng)}}
\end{center}

\vspace{1cm}

\noindent\textit{Hôm nay, ngày \textbf{04} tháng \textbf{02} năm \textbf{2026}, tại Trường THPT Phước Long, chúng tôi gồm:}

\vspace{0.5cm}

% ============ BÊN A ============
\section*{\textcolor{schoolblue}{BÊN A: TRƯỜNG TRUNG HỌC PHỔ THÔNG PHƯỚC LONG}}
\begin{tabular}{p{5cm}p{10cm}}
\textbf{Tên trường:} & Trường Trung học Phổ thông Phước Long \\
\textbf{Tên giao dịch:} & Phuoc Long High School \\
\textbf{Địa chỉ:} & 18D Dương Đình Hội, Phường Phước Long B, Thủ Đức, TP. Hồ Chí Minh \\
\textbf{Điện thoại:} & 028 3731 1996 \\
\textbf{Email:} & thptphuoclong@hcm.edu.vn \\
\textbf{Website:} & \href{https://thptphuoclong.hcm.edu.vn}{thptphuoclong.hcm.edu.vn} \\
\textbf{Mã số thuế:} & 4200123456 \\
\textbf{Đại diện:} & Ông Trần Văn Minh \\
\textbf{Chức vụ:} & Hiệu trưởng \\
\textbf{Tài khoản:} & 1234567890 -- Ngân hàng Vietcombank Chi nhánh TP. Hồ Chí Minh \\
\end{tabular}

\vspace{0.8cm}

% ============ BÊN B ============
\section*{\textcolor{schoolblue}{BÊN B: GIÁO VIÊN}}
\begin{tabular}{p{5cm}p{10cm}}
\textbf{Họ và tên:} & Nguyễn Phương Thảo \\
\textbf{Ngày sinh:} & 15/08/1995 \\
\textbf{Giới tính:} & Nữ \\
\textbf{Số CMND/CCCD:} & 023095012345 \\
\textbf{Ngày cấp:} & 10/01/2020, tại Công an TP. Hồ Chí Minh \\
\textbf{Nơi sinh:} & Thành phố Hồ Chí Minh \\
\textbf{Hộ khẩu thường trú:} & 123 Nguyễn Văn Linh, Quận 7, TP. Hồ Chí Minh \\
\textbf{Chỗ ở hiện tại:} & 456 Lê Lợi, Phường Phước Long B, Thủ Đức, TP. Hồ Chí Minh \\
\textbf{Điện thoại:} & 0912.345.678 \\
\textbf{Email:} & \href{mailto:phuongthao@thptphuoclong.hcm.edu.vn}{phuongthao@thptphuoclong.hcm.edu.vn} \\
\textbf{Trình độ chuyên môn:} & Thạc sĩ Ngôn ngữ Anh -- Đại học Sư phạm TP.HCM \\
\textbf{Chứng chỉ nghiệp vụ:} & TESOL Certificate, IELTS 8.0 \\
\textbf{Số tài khoản:} & 9876543210 -- Ngân hàng Techcombank \\
\end{tabular}

\vspace{0.5cm}

\noindent\textit{Sau khi nghiên cứu nhu cầu, khả năng và nguyện vọng của hai bên, căn cứ vào Bộ luật Lao động số 45/2019/QH14 và các quy định pháp luật hiện hành, hai bên thỏa thuận ký kết hợp đồng lao động với các điều khoản như sau:}

\newpage

% ============ ĐIỀU 1 ============
\section*{\textcolor{schoolblue}{ĐIỀU 1: CÔNG VIỆC VÀ ĐỊA ĐIỂM LÀM VIỆC}}

\subsection*{1.1. Vị trí công việc}
Bên A tuyển dụng Bên B làm \textbf{Giáo viên Tiếng Anh} với các nhiệm vụ cụ thể như sau:

\begin{itemize}
    \item Giảng dạy môn Tiếng Anh cho học sinh lớp 10, 11, 12
    \item Biên soạn giáo án, tài liệu giảng dạy phù hợp với chương trình
    \item Tham gia công tác chủ nhiệm lớp (nếu được phân công)
    \item Tham gia các hoạt động giáo dục, ngoại khóa của nhà trường
    \item Tham gia bồi dưỡng học sinh giỏi, phụ đạo học sinh yếu kém
    \item Các công việc khác theo sự phân công của Hiệu trưởng
\end{itemize}

\subsection*{1.2. Địa điểm làm việc}
Trường THPT Phước Long, 18D Dương Đình Hội, Phường Phước Long B, Thủ Đức, TP. Hồ Chí Minh

\subsection*{1.3. Thời gian làm việc}
\begin{itemize}
    \item Thời gian làm việc: 40 giờ/tuần, từ Thứ 2 đến Thứ 6
    \item Ca làm việc: 07:00 -- 11:30 và 13:30 -- 17:00
    \item Nghỉ: Thứ 7, Chủ nhật và các ngày lễ, tết theo quy định
\end{itemize}

% ============ ĐIỀU 2 ============
\section*{\textcolor{schoolblue}{ĐIỀU 2: THỜI HẠN HỢP ĐỒNG}}

\subsection*{2.1. Thời hạn}
Hợp đồng có thời hạn \textbf{12 tháng}, kể từ ngày \textbf{01/03/2026} đến ngày \textbf{28/02/2027}.

\subsection*{2.2. Thời gian thử việc}
Thời gian thử việc: \textbf{60 ngày}, từ ngày 01/03/2026 đến ngày 29/04/2026.

Trong thời gian thử việc, mỗi bên có quyền đơn phương chấm dứt hợp đồng bằng thông báo trước ít nhất 03 ngày làm việc.

% ============ ĐIỀU 3 ============
\section*{\textcolor{schoolblue}{ĐIỀU 3: MỨC LƯƠNG VÀ CHẾ ĐỘ ĐÃI NGỘ}}

\subsection*{3.1. Mức lương}
\begin{table}[h!]
\renewcommand{\arraystretch}{1.3}
\begin{tabular}{|p{8cm}|r|}
\hline
\rowcolor{lightgray}
\textbf{Khoản mục} & \textbf{Số tiền (VNĐ)} \\
\hline
Lương cơ bản (theo hệ số 3.5, mức lương tối thiểu vùng IV) & 8,500,000 \\
\hline
Phụ cấp trách nhiệm giáo viên & 1,200,000 \\
\hline
Phụ cấp ưu đãi theo nghề & 800,000 \\
\hline
Phụ cấp vùng (Vùng khó khăn) & 500,000 \\
\hline
Phụ cấp chứng chỉ ngoại ngữ & 2,000,000 \\
\hline
\rowcolor{schoolblue!20}
\textbf{Tổng lương tháng} & \textbf{13,000,000} \\
\hline
\end{tabular}
\end{table}

\noindent\textit{Lương chưa bao gồm các khoản BHXH, BHYT, BHTN, thuế TNCN theo quy định pháp luật.}

\subsection*{3.2. Lương làm thêm giờ}
Giáo viên được hưởng thù lao giảng dạy thêm giờ (nếu có):
\begin{itemize}
    \item Lương giờ chuẩn: 180,000 VNĐ/tiết (45 phút)
    \item Lương giờ phụ đạo học sinh: 150,000 VNĐ/tiết
    \item Lương giờ bồi dưỡng học sinh giỏi: 200,000 VNĐ/tiết
\end{itemize}

\subsection*{3.3. Chế độ thưởng}
\begin{itemize}
    \item Thưởng chuyên cần: 500,000 VNĐ/tháng (nếu không nghỉ phép)
    \item Thưởng theo hiệu quả công việc: Tối đa 2,000,000 VNĐ/tháng
    \item Thưởng tết Nguyên đán: Theo quy định của nhà trường
    \item Thưởng học sinh đạt giải: Theo quy chế khen thưởng
\end{itemize}

\subsection*{3.4. Hình thức thanh toán}
\begin{itemize}
    \item Thanh toán qua tài khoản ngân hàng
    \item Thời gian: Trước ngày 10 của tháng tiếp theo
\end{itemize}

\subsection*{3.5. Các chế độ phúc lợi khác}
\begin{itemize}
    \item Được tham gia BHXH, BHYT, BHTN đầy đủ theo luật định
    \item Hỗ trợ nhà ở: 3,000,000 VNĐ/tháng
    \item Hỗ trợ đi lại: 800,000 VNĐ/tháng
    \item Nghỉ phép năm: 12 ngày/năm (hưởng nguyên lương)
    \item Khám sức khỏe định kỳ: 01 lần/năm
    \item Tham gia các hoạt động văn hóa, thể thao, du lịch của nhà trường
    \item Hỗ trợ đào tạo, bồi dưỡng nâng cao trình độ chuyên môn
\end{itemize}

% ============ ĐIỀU 4 ============
\section*{\textcolor{schoolblue}{ĐIỀU 4: QUYỀN VÀ NGHĨA VỤ CỦA BÊN A}}

\subsection*{4.1. Quyền của Bên A}
\begin{enumerate}
    \item Yêu cầu Bên B thực hiện đúng các nhiệm vụ đã giao theo hợp đồng
    \item Kiểm tra, đánh giá chất lượng giảng dạy của Bên B
    \item Thu hồi, điều chuyển công việc phù hợp với năng lực của Bên B
    \item Khen thưởng hoặc kỷ luật Bên B theo quy chế của nhà trường
    \item Đơn phương chấm dứt hợp đồng theo quy định pháp luật
\end{enumerate}

\subsection*{4.2. Nghĩa vụ của Bên A}
\begin{enumerate}
    \item Trả lương và các chế độ đãi ngộ đầy đủ, đúng hạn
    \item Tạo điều kiện làm việc thuận lợi, cung cấp trang thiết bị giảng dạy
    \item Tham gia BHXH, BHYT, BHTN cho Bên B theo luật định
    \item Đào tạo, bồi dưỡng nâng cao trình độ chuyên môn cho Bên B
    \item Bảo vệ quyền và lợi ích hợp pháp của Bên B
    \item Tạo điều kiện để Bên B nghỉ phép, nghỉ lễ, tết theo quy định
\end{enumerate}

% ============ ĐIỀU 5 ============
\section*{\textcolor{schoolblue}{ĐIỀU 5: QUYỀN VÀ NGHĨA VỤ CỦA BÊN B}}

\subsection*{5.1. Quyền của Bên B}
\begin{enumerate}
    \item Được hưởng lương và các chế độ đãi ngộ theo hợp đồng
    \item Được làm việc trong môi trường an toàn, đầy đủ trang thiết bị
    \item Được đào tạo, bồi dưỡng nâng cao trình độ nghề nghiệp
    \item Được nghỉ phép, nghỉ lễ, tết theo quy định
    \item Được tham gia quản lý, đóng góp ý kiến xây dựng nhà trường
    \item Được giải quyết khiếu nại, khúc mắc theo quy định
    \item Đơn phương chấm dứt hợp đồng theo quy định pháp luật
\end{enumerate}

\subsection*{5.2. Nghĩa vụ của Bên B}
\begin{enumerate}
    \item Thực hiện đầy đủ nhiệm vụ được giao, đảm bảo chất lượng giảng dạy
    \item Chấp hành nghiêm chỉnh nội quy, quy chế của nhà trường
    \item Giữ gìn đạo đức nhà giáo, tác ph풍 sư phạm
    \item Không tiết lộ bí mật, thông tin nhạy cảm của nhà trường
    \item Bảo quản tài sản, thiết bị được giao
    \item Tham gia đầy đủ các hoạt động tập thể của nhà trường
    \item Thông báo kịp thời khi nghỉ việc đột xuất
    \item Báo cáo kết quả công việc theo định kỳ
\end{enumerate}

% ============ ĐIỀU 6 ============
\section*{\textcolor{schoolblue}{ĐIỀU 6: BẢO MẬT THÔNG TIN}}

\subsection*{6.1. Cam kết bảo mật}
Bên B cam kết không tiết lộ các thông tin bảo mật sau đây:
\begin{itemize}
    \item Tài liệu, giáo trình, phương pháp giảng dạy của nhà trường
    \item Thông tin về học sinh, phụ huynh
    \item Chiến lược phát triển, kế hoạch kinh doanh của nhà trường
    \item Các thông tin khác được xác định là bảo mật
\end{itemize}

\subsection*{6.2. Thời gian bảo mật}
Nghĩa vụ bảo mật có hiệu lực trong suốt thời gian làm việc và sau khi chấm dứt hợp đồng 24 tháng.

% ============ ĐIỀU 7 ============
\section*{\textcolor{schoolblue}{ĐIỀU 7: CHẤM DỨT HỢP ĐỒNG}}

\subsection*{7.1. Hợp đồng chấm dứt trong các trường hợp sau:}
\begin{enumerate}
    \item Hết thời hạn hợp đồng mà không gia hạn
    \item Hai bên thỏa thuận chấm dứt hợp đồng
    \item Bên B chết, mất tích, mất năng lực hành vi dân sự
    \item Các trường hợp khác theo quy định tại Điều 34, 35, 36 Bộ luật Lao động
\end{enumerate}

\subsection*{7.2. Đơn phương chấm dứt hợp đồng}
\textbf{Bên A có quyền đơn phương chấm dứt hợp đồng khi:}
\begin{itemize}
    \item Bên B vi phạm kỷ luật lao động nghiêm trọng
    \item Bên B không hoàn thành công việc 02 tháng liên tiếp
    \item Bên B bị ốm đau, tai nạn đã điều trị 12 tháng liên tiếp
    \item Các trường hợp khác theo Điều 36 Bộ luật Lao động
\end{itemize}

\textbf{Bên B có quyền đơn phương chấm dứt hợp đồng khi:}
\begin{itemize}
    \item Không được bố trí theo đúng công việc, địa điểm đã thỏa thuận
    \item Không được trả lương đầy đủ, đúng hạn
    \item Bị ngược đãi, cưỡng bức lao động
    \item Các trường hợp khác theo Điều 35 Bộ luật Lao động
\end{itemize}

Khi đơn phương chấm dứt hợp đồng, phải báo trước cho bên kia ít nhất 30 ngày.

% ============ ĐIỀU 8 ============
\section*{\textcolor{schoolblue}{ĐIỀU 8: GIẢI QUYẾT TRANH CHẤP}}

\subsection*{8.1. Thương lượng}
Mọi tranh chấp phát sinh từ hợp đồng này sẽ được giải quyết trước tiên thông qua thương lượng, hòa giải.

\subsection*{8.2. Trọng tài hoặc Tòa án}
Nếu không đạt được thỏa thuận trong vòng 30 ngày, tranh chấp sẽ được đưa ra Tòa án nhân dân có thẩm quyền tại TP. Hồ Chí Minh giải quyết theo quy định pháp luật.

% ============ ĐIỀU 9 ============
\section*{\textcolor{schoolblue}{ĐIỀU 9: ĐIỀU KHOẢN CUỐI CÙNG}}

\subsection*{9.1. Hiệu lực}
Hợp đồng này có hiệu lực kể từ ngày \textbf{01/03/2026}.

\subsection*{9.2. Số bản}
Hợp đồng được lập thành \textbf{02 bản} có giá trị pháp lý như nhau, mỗi bên giữ 01 bản.

\subsection*{9.3. Sửa đổi, bổ sung}
Mọi sửa đổi, bổ sung hợp đồng phải được lập thành văn bản và có chữ ký của hai bên.

\subsection*{9.4. Cam kết}
Hai bên cam kết thực hiện đúng và đầy đủ các điều khoản của hợp đồng. Bản hợp đồng đã được đọc kỹ, hiểu rõ nội dung trước khi ký.

\vspace{1.5cm}

% ============ CHỮ KÝ ============
\begin{center}
\textit{Thành phố Hồ Chí Minh, ngày 04 tháng 02 năm 2026}
\end{center}

\vspace{1cm}

\begin{tabular}{p{7.5cm}p{7.5cm}}
\centering
\textbf{\textcolor{schoolblue}{ĐẠI DIỆN BÊN A}} & 
\centering
\textbf{\textcolor{schoolblue}{ĐẠI DIỆN BÊN B}} \\[0.2cm]
\centering
\textbf{HIỆU TRƯỞNG} &
\centering
\textbf{GIÁO VIÊN} \\[0.2cm]
\centering
\textit{(Ký, ghi rõ họ tên và đóng dấu)} &
\centering
\textit{(Ký và ghi rõ họ tên)} \\[3.5cm]
\centering
\textbf{Trần Văn Minh} &
\centering
\textbf{Nguyễn Phương Thảo} \\
\end{tabular}

\vspace{1cm}

\begin{center}
    \textcolor{schoolblue}{\rule{\textwidth}{1pt}}\\[0.3cm]
    \textit{\small Hợp đồng này là cơ sở pháp lý để thực hiện quyền và nghĩa vụ của hai bên.\\
    Mọi thắc mắc xin liên hệ: Phòng Tổ chức -- Hành chính | ĐT: 028 3731 1996 | Email: thptphuoclong@hcm.edu.vn}
\end{center}

\end{document}
